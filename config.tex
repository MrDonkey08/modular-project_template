\documentclass[10pt,twocolumn,letterpaper]{article}

%% Language and font encodings
\usepackage[spanish]{babel}
\usepackage[utf8]{inputenc}
\usepackage[T1]{fontenc}
\usepackage{times} % Times New Roman
\usepackage{courier}

%% Sets page size and margins
\usepackage[top=0.75in,bottom=1in,left=0.68in,right=0.68in]{geometry}
\setlength{\columnsep}{0.17in} % page columns separation

%% Useful packages
\usepackage{amsmath}
\usepackage{array} % <-- add this line for m{} column type
\usepackage[hidelinks]{hyperref} % hyperlinks support
\usepackage{float} % to insert images at same place as latex code with H
\usepackage{graphicx} % images support
%\usepackage{listings} % codeblock support
% \usepackage{longtable} % longtable support
%\usepackage{smartdiagram} % diagrams support
\usepackage[most]{tcolorbox} % callouts support
%\usepackage[colorinlistoftodos]{todonotes}
\usepackage[dvipsnames, table, xcdraw]{xcolor} % Tables support
% \usepackage{tabularx}
%\usepackage{zed-csp} % cchemas support

%% Formating
\usepackage{authblk} % to add authors in maketitle
%\usepackage{blindtext} % to gen filler text
\usepackage[labelfont=footnotesize, labelformat=simple, labelsep=period, figurename=Fig.]{caption} % to change prefix of the image caption
% \usepackage{apacite}
\usepackage{cite} % useful to compress multiple quotations into a single entry
\usepackage{enumitem}
\usepackage{etoolbox} % for \patchcmd
\usepackage{fancyhdr} % to set page style
\usepackage[none]{hyphenat} % Disables hyphenation
\usepackage{indentfirst}
%\usepackage{natbib}
%\usepackage{parskip} % remove first line tabulation
\usepackage{setspace}
\usepackage{titlesec}
%\usepackage{titling} % to config maketitle

%% Variables
% Main images
\newcommand{\logoUdg}{logo-udg.jpg}
\newcommand{\logoCucei}{logo-cucei.jpg}

% School data
\newcommand{\universidad}{Universidad de Guadalajara}
\newcommand{\cede}{Centro Universitario de Ciencias Exactas e Ingenierías}

% Subject data
\newcommand{\materia}{Materia}
\newcommand{\carrera}{Ingeniería en Computación}
\newcommand{\division}{División de Tecnologías para la Integración CiberHumana}
\newcommand{\theTitle}{Formato para Presentar Proyectos Modulares}
\newcommand{\profesor}{Profesor}
\newcommand{\seccion}{Sección}
\newcommand{\nrc}{NRC}
\newcommand{\clave}{Clave}
\newcommand{\generation}{2025A}
\newcommand{\startDate}{12 de mayo de 2025}

% Author(s) data
\newcommand{\theAuthor}{Nombre del líder del proyecto}
\newcommand{\bAuthor}{nombre del segundo participante}
\newcommand{\cAuthor}{nombre del tercer participante}
\newcommand{\dAuthor}{nombre del asesor}

\newcommand{\theAuthorCode}{A code}
\newcommand{\bAuthorCode}{B code}
\newcommand{\cAuthorCode}{C code}
\newcommand{\dAuthorCode}{D code}

\newcommand{\theAuthorMail}{first@mail.com}
\newcommand{\bAuthorMail}{second@mail.com}
\newcommand{\cAuthorMail}{third@mail.com}
\newcommand{\dAuthorMail}{fourth@mail.com}

% Authors titles
\newcommand{\theAuthorTitle}{Ingeniero en Computación}
\newcommand{\bAuthorTitle}{Ingeniero en Computación y Electrónica}
\newcommand{\cAuthorTitle}{Ingeniero en Biomédica}

% Repo data
\newcommand{\repositorio}{}
\newcommand{\version}{}
\newcommand{\licencia}{}

% Team data
\newcommand{\teamName}{Equipo}

% Content
\newcommand{\moduloText}{
	Se describe brevemente la relación que existe de este módulo con el proyecto
	modular a presentar, usar como referencia el siguiente enlace.
	\url{http://www.cucei.udg.mx/carreras/computacion/sites/default/files/adjuntos/criteriosaprobacion_0.pdf}
}

%% Declaration
\date{}
\graphicspath{ {img/} }
\addto\captionsspanish{\renewcommand{\contentsname}{Índice}}
\renewcommand{\lstlistingname}{Código} % Para cambiar el caption de los código

%% Fonts sizes
\makeatletter
\renewcommand{\tiny}{\fontsize{6.5pt}{7pt}\selectfont} % 6pt with 7pt line spacing
\renewcommand{\scriptsize}{\fontsize{7pt}{8pt}\selectfont} % 7pt / 8pt
\renewcommand{\footnotesize}{\fontsize{8pt}{9pt}\selectfont} % 8pt / 9pt
\renewcommand{\small}{\fontsize{9pt}{10pt}\selectfont} % 9pt / 10pt
\renewcommand{\normalsize}{\fontsize{10pt}{11pt}\selectfont} % 10pt / 11pt (default)
\renewcommand{\large}{\fontsize{11pt}{12pt}\selectfont} % 11pt / 12pt
\renewcommand{\Large}{\fontsize{12pt}{14pt}\selectfont} % 12pt / 14pt
\renewcommand{\LARGE}{\fontsize{14pt}{16pt}\selectfont} % 14pt / 16pt
\renewcommand{\huge}{\fontsize{16pt}{18pt}\selectfont} % 16pt / 18pt
\renewcommand{\Huge}{\fontsize{24pt}{26pt}\selectfont} % 24pt / 26pt
\makeatother

\AtBeginEnvironment{thebibliography}{\footnotesize}	% or \footnotesize

%% Spacing
\newcommand{\nl}{\par\vspace{0.4cm}}
% \setlength{\parskip}{10.5pt} % Espaciado de línea anterior
\setstretch{1.03} % Interlineado
%\setlength{\parindent}{4pt} % Sangría

% Itemize and enumerate
\setlist[itemize]{itemsep=0pt, topsep=0pt, parsep=1.08pt, partopsep=1pt, after=\vspace{10.45pt}}
\setlist[enumerate]{itemsep=0pt, topsep=0pt, parsep=1.08pt, partopsep=1pt, after=\vspace{10.45pt}}

% Captions
\setlength{\floatsep}{12pt}
\setlength{\intextsep}{12pt}

% Tables
\let\oldtabular\tabular
\renewcommand{\tabular}{\footnotesize\oldtabular}
% \renewcommand{\arraystretch}{1.2} % <-- Adjust vertical spacing
\addto\captionsspanish{\renewcommand{\tablename}{\scshape \footnotesize Tabla}}

% --- Set caption spacing to 0pt for tables ---
% \captionsetup[table]{skip=0pt}

%% Styles
\hypersetup{colorlinks=false} % Esilo de enlaces
\urlstyle{same}

\pagestyle{fancy}
% Header
\renewcommand{\headrulewidth}{0pt} % Elimina la línea del encabezado
% \setlength{\headheight}{15pt} % Ajuste necesario para evitar warnings

\fancyhead{}
\rhead{
	\begin{flushright}
		\theTitle
	\end{flushright}
}

% Footer
\fancyfoot{}
\lfoot{\small\materia}
\cfoot{\thepage} % Paginación
\rfoot{\small Curso impartido por \profesor}

%% Title
\makeatletter
\renewcommand{\@maketitle}{%
	\newpage
	\null
	\vskip 0em%
	\begin{center}
		\begin{minipage}{0.7\textwidth}
			\centering
			\let \footnote \thanks
			{\LARGE \@title}
			% \vskip 1.5em%
			{\large
				\lineskip .5em%
				\begin{tabular}[t]{c}
					\@author
				\end{tabular}
			}
			\vskip -.5em%
			{\university}
			\vskip .5em%
				{\small \mails}
			% \vskip .5em%
			% {\large \@date}
			% \vskip 1em%
		\end{minipage}
	\end{center}
}
\makeatother

\title{
	\Huge \theTitle
}

\author{\theAuthor}
\author{\bAuthor}
\author{\cAuthor}
\author{\dAuthor}

\newcommand{\university}{
	\textit{CENTRO UNIVERSITARIO DE CIENCIAS}\\
	\textit{EXACTAS E INGENIERÍAS, (CUCEI, UDG)}
}

\newcommand{\mails}{
	\fontfamily{pcr}\selectfont
	\theAuthorMail\\
	\fontfamily{pcr}\selectfont
	\bAuthorMail\\
	\fontfamily{pcr}\selectfont
	\cAuthorMail\\
	\fontfamily{pcr}\selectfont
	\dAuthorMail
}

\affil{}
